% !TEX root = seminararbeit.tex

\section{Results}\label{results}

% Die vorliegende Arbeit beschäftigt sich mit der iterativen Entwicklung einer Augmented Reality Anwendung für das Android-Betriebssystem. Ziel dabei war es eine Anwendung zu entwerfen, die als Vorlage für die Generierung mittels der SSIML-Sprachfamilie dient. Dafür wurden ver- schiedene bereits vorhandene Frameworks auf ihre Tauglichkeit für diesen Zweck überprüft. Da- nach wurde ein High-Level Framework für Android entworfen, mit welchem es möglich ist durch minimalem Programmieraufwand eine Augmented Reality Anwendung zu erzeugen. Dieses Fra- mework dient als Adapter zu low-level Frameworks und reduziert den vom Entwickler benötigten Quellcode auf ein Minimum, wodurch eine Codegenerierung mittels domänenspezifischer Spra- chen gut umgesetzt werden kann. Leider war es nicht möglich bestimmte Eigenschaften einer Augmented Reality Anwendung umzusetzen da sie den Rahmen dieser Arbeit sprengen würde. Das Überlagern von 3D-Objekte durch ein reales Objekt, oder ein sich nach der Umgebungs- beleuchtung anpassendes Lichtsystem gehören hier zu. Außerdem wäre es möglich das Sensor System um Kollisionssensoren oder Annäherungssensoren zu erweitern. Abschließend kann fest- gestellt werden, das Hilfe des in dieser Arbeit vorgestellte Augmented Reality Framework es mög- lich ist eine performante AR-Anwendung ohne große Probleme mittels der SSIML-Sprachfamilie zu Erzeugen.

\subsection{Why Scegratoo is the best since sliced
Bread}\label{why-scegratoo-is-the-best-since-sliced-bread}
