% !TEX root = seminararbeit.tex

\section{Implementation}
\label{implementation}

\subsection{Server}
\label{used-tech}

\subsubsection{Koa}
\label{koa}

The server has a really small interface.
When it gets a request it tries to to match the request to one of its routes, otherwise tries to serve the requested file from the file system (assuming it is a static file), otherwise answers with a 404, which means \texttt{Not Found}.

Some of the routes it answers to are:
\begin{description*}
  \item[POST /projects/:project/src/:file]
    saving files being uploaded
  \item[GET /projects]
    return all projects on the server
  \item[GET /projects/:project]
    return all information for the project \texttt{:project}
\end{description*}

\subsection{Client}
\label{used-tech}

\subsubsection{angular}
\label{angular}

AngularJS is used for:

\begin{itemize*}
  \item bootstrapping the application,
  \item modularization,
  \item dependency management,
  \item resource management,
  \item routing and
  \item less dynamic views.
\end{itemize*}

\paragraph{Bootstrapping}
\label{par:Bootstrapping and Routing}

Listing \ref{list:bootstrap} initializes an SceGraToo. In the \texttt{config}
function routes are defined. When a link is clicked the browser will not make a
request to the server and load that page instead a new controller takes over and
renders a different templates. The advantage of this approach is that the user
sees immediate feedback while navigating. The application can render the view and
react to user input while its still waiting for some requested resources from
the server (like a list of all available projects). This approach is called
single-page application \cite{Mikowski:2013:SPW:2663433}.

\begin{listing}
  \begin{minted}[breaklines,bgcolor=bg]{javascript}
window.angular.module('scegratooApp')
  .config(function ($routeProvider) {
    $routeProvider
      .when('/', {
        redirectTo: '/projects'
      })
      .when('/projects', {
        templateUrl: 'views/projects.html',
        controller: 'ProjectsCtrl'
      })
      .when('/projects/:project', {
        templateUrl: 'views/project.html',
        controller: 'ProjectCtrl'
      })
      .when('/projects/:project/:file*', {
        templateUrl: 'views/projects/:project/x3d/:file.html',
        controller: 'ProjectsProjectX3dFileCtrl'
      })
  })
  \end{minted}
  \caption{This is how SceGraToo is initialized. It also shows how the routing is defined.}
  \label{list:bootstrap}
\end{listing}

\paragraph{Modularisation and Dependency Injection}
\label{par:modularisation}

Each controller, view or service is contained in its own module and does not
pollute the global name space. In a browser's javascript context the global
name space refers to the name space that belongs to the \texttt{window} object.
Defining variables in scripts defines this variable on the window object.
Angular modules prevent this. Modules are registered on a specific angular
application (thus one website could also accommodate multiple angular
applications). The defined variables are contained by creating function that
returns whatever the module is supposed to contain, thus creating a closure.
Listing \ref{list:angularmodule} shows a module that creates a WeakMap and
returns it. Modules can denote that they depend on other modules, this can be
seen in listing \ref{list:depinj}. The \texttt{MoveableUtils} request that the
\texttt{moveable} module is injected into it when initializing. Angular creates
a dependency graph and resolves dependencies automatically.


\begin{listing}
  \begin{minted}[breaklines,bgcolor=bg]{javascript}
window.angular.module('scegratooApp')
  .service('moveables', function () {
    return new WeakMap()
  })
  \end{minted}
  \caption{This module creates a WeakMap that can be injected in multiple other modules. These modules all share the same WeakMap since services are singletons. \texttt{service}'s first argument is the \texttt{service}'s name, that can be used by other modules to import it.}
  \label{list:angularmodule}
\end{listing}

\begin{listing}
  \begin{minted}[breaklines,bgcolor=bg]{javascript}
angular.module('scegratooApp')
  .service('MoveableUtils', function (moveables) {
    return {
      logMoveables: () => console.log(moveables)
    }
  })
  \end{minted}
  \caption{This module requests the \texttt{moveables} module to be injected.}
  \label{list:depinj}
\end{listing}

\paragraph{Views}
\label{par:Views}

Angular tempates are mostly logicless
(except for ng-repeat iterators and ng-if statements). Listing \ref{list:view}
shows a template that renders all prejects the corresponding controller
retrieved from the server.

\begin{listing}
  \begin{minted}[breaklines,bgcolor=bg]{html}
<sgt-navigation-bar>
</sgt-navigation-bar>
<div class="sash">
  <h3>
    Editable files for {{project.name}}
  </h3>
  <div ng-repeat="file in project.files | orderBy:'view'">
    <div ng-show="file.view">
      {{file.view}} -
      <a
        href="#/projects/{{projectName}}/ {{file.view}}/{{file.path}}">
        {{file.path}}
      </a>
    </div>
  </div>
</div>
  \end{minted}
  \caption{A template that renders projects that the controller retrieved from the server}
  \label{list:view}
\end{listing}

\subsubsection{react}
\label{react}

React is utilized by SceGraToo to render the tree-view that gives a more
structured view of the scene-graph than the rendered scene does.

React works by creating components and nesting them. Listing
\ref{list:reactcomponent} shows the \texttt{TreeView} component. The
\texttt{TreeNode} is another component that handles a specific tree-view-node,
components keep instantiating and returning components until the whole
scene-graph is traversed. In listing \ref{list:reactentrypoint} it is shown how
it is rendered to the DOM. The html syntax is simply syntactic sugar and is
transpiled into normal javascript before being evaluated (the transpiled
equivalent of listing \ref{list:reactentrypoint} is shown in listing
\ref{list:reacttranspiled}).

Using react the view virtually becomes a function of its input. The input is the root node of the scene-graph, the x3d node.

The parsing and rendering process can be described as follows:
\begin{enumerate*}
  \item Choose a graph node as the root
  \item call the node component with that graph node,
  \item instantiate corresponding components for each of the nodes attribute,
  \item if the graph node has child nodes call the node component again with each child node and return their return values or
  \item if the graph node has no children return an empty element.
\end{enumerate*}

\begin{listing}
  \begin{minted}[breaklines,bgcolor=bg]{javascript}
React.createClass({
  displayName: 'TreeView',
  propTypes: {
    data: React.PropTypes.object.isRequired
  },
  render: function () {
    if (this.props.data.runtime) {
      return (
        <TreeNode
          data={this.props.data.querySelector('scene')}
          runtime={this.props.data.runtime}
        />
      )
    } else {
      return <div/>
    }
  }
})
  \end{minted}
  \caption{The TreeView component is instantiated with a node. Its render function returns an instantiated TreeNode unless the given node has no runtime property, in that case it just returns an empty div.}
  \label{list:reactcomponent}
\end{listing}

\begin{listing}
  \begin{minted}[breaklines,bgcolor=bg]{javascript}
const treeViewContainer = document.querySelector('#container')
const x3dNode = document.querySelector('x3d')
React.render(<TreeView data={x3dNode} />, treeViewContainer)
  \end{minted}
  \caption{}
  \label{list:reactentrypoint}
\end{listing}

\begin{listing}
  \begin{minted}[breaklines,bgcolor=bg]{javascript}
const treeViewContainer = document.querySelector('#container')
const x3dNode = document.querySelector('x3d')
React.render(React.createElement(TreeView, { data: x3dNode }), treeViewContainer)
  \end{minted}
  \caption{}
  \label{list:}
\end{listing}

\subsubsection{Synchronization Process}
\label{synchronization-process}

Synchronizing the tree-view when the scene-graph changes is done by calling
\texttt{React.render} again, just like in listing \ref{reactentrypoint}. React
calculates the changes that need to be done to update the DOM and applies these.

If the tree-view changes the scene-graph, the same thing happens. Listing
\ref{list:checkbox} show a check box component. It receives an property called
\texttt{owner}. That is a scene-graph node. Nodes in x3dom have the render
property. If the property is true, that node and all its children are rendered,
if not, they are not visible. The component is showing the state of the the
\texttt{owner}'s render property's state. When the user clicks that check box the
\texttt{owner}'s attribute is changed. The component does not have to update the
DOM node, react is doing it the next time \texttt{React.render} is called. This
is a simple example but the concept holds up for more complicated interactions
like adding new nodes or moving via drag 'n drop.

\begin{listing}
  \begin{minted}[breaklines,bgcolor=bg]{javascript}
const TreeNodeAttributeRender = React.createClass({
  displayName: 'TreeNodeAttributeRender',
  propTypes: {
    owner: React.PropTypes.object.isRequired,
  },
  changeHandler: function (event) {
    if (event.currentTarget.checked) {
      this.props.owner.setAttribute('render', true)
    } else {
      this.props.owner.setAttribute('render', false)
    }
  },
  render: function () {
    const attribute = this.props.owner.getAttribute('render')
    const checked = render === 'true'

    return <input type='checkbox' checked={checked} onChange={this.changeHandler} />
  }
})
  \end{minted}
  \caption{A component that renders a checkbox that show the \texttt{owner} render property's state. Clicking the checkbox changes the \texttt{owner}'s property's state.}
  \label{list:checkbox}
\end{listing}
