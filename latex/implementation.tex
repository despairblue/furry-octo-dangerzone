% !TEX root = seminararbeit.tex

\section{Implementation}\label{implementation}

\subsection{Used Tech}\label{used-tech}

\paragraph{angular}\label{angular}

\begin{itemize*}
\item
  dependency management
\item
  resource management
\item
  less dynamic templates
\end{itemize*}

\paragraph{react}\label{react}

\begin{itemize*}
\item
  highly dynamic views (like the tree view)
\end{itemize*}

\subsection{Problems}\label{problems}

\paragraph{synchronization Process}\label{synchronization-process}

This design leads to brittle code that is hard to maintain and hard to
adapt to new use cases.

The first culrpit in this design is that the tree-view-controller
mutates itself.

Let's externalize the mutation observation and update capabilities into
another actor: a mutation observer.

The mutation observer can map each graph node to its corresponding tree
node and thus change the tree node to represent the matching graph node
again.

The tree-view-controller does not mutate itself anymore. All the
mutation logic that synchronizes the tree-view-controller with the
scene-graph is in the mutation observer. But the scene-graph and the
tree-view-controller can still diverge, since moving the mutation logic
from the tree-view-controller to the mutation observer does not make it
easier to reason about and thus less error prone.

Assuming that the first operation (traversing the scene-graph and
creating the tree-view-controller) is correctly implemented and
resilient, the easiest way to make sure the tree-view-controller and the
scene-graph are in sync would be to recreate the tree-view-controller
whenever the scene-graph changes.

At first sight that seems to be awfully slow, that needn't to be true.
Instead of recreating the the whole tree using {[}DOM{]} elements, a
model could be created using simple javascript objects. This virtual
tree-view-controller can then be compared to the tree-view-controller
already in the {[}DOM{]} and only the changes need to be applied.

That might sound like it's no easier than doing the synchronization
manually like in the first diagram, but the diff algorithm doesn't
actually need to know what it's diffing. Meaning it could be developed
and tested once and be used by all kinds of projects. That's one thing
react provides. The rendering pipeline looks finally like this:\\

React calls the virtual representation of what will be rendered
\emph{virtual DOM}. The other important feature of react is it's way to
build the virtual DOM. A is a factory that returns A virtual DOM node is
the return value of component's render function, the render function can
nest other virtual {[}DOM{]} nodes in its return value.

A quick examples:

\begin{minted}[breaklines,bgcolor=bg]{javascript}
  // TreeNodeAttributeList :: [Attribute] -> div
  var TreeNodeAttributeList = React.createClass({
    render: function () {
      // the scene-graph node that was passed `createElement` by the caller
      var node = this.props.node;

      var whitelist = ['def', 'diffusecolor', 'orientation', 'position', 'render', 'rotation', 'scale', 'translation', 'url'];

      var attributesToRender = node.attributes.filter(function (attribute) {
        return propsToRender.includes(attribute.name.toLowerCase());
      })

      return (
        <div>
          {
            attributesToRender.map(function (attribute) {
              return <TreeNodeAttribute attribute={a} owner={node}/>
            })
          }
        </div>
      );
    }
  });
\end{minted}

The usage of HTML tags is just syntactic sugar, it's transpiled into:

\begin{minted}[breaklines,bgcolor=bg]{javascript}
  'use strict';

  // TreeNodeAttributeList :: [Attribute] -> div
  var TreeNodeAttributeList = React.createClass({
    render: function render() {
      // the scene-graph node that was passed `createElement` by the caller
      var node = this.props.node;

      var whitelist = ['def', 'diffusecolor', 'orientation', 'position', 'render', 'rotation', 'scale', 'translation', 'url'];

      var attributesToRender = node.attributes.filter(function (attribute) {
        return propsToRender.includes(attribute.name.toLowerCase());
      });

      return React.createElement(
        'div',
        null,
        attributesToRender.map(function (attribute) {
          return React.createElement(TreeNodeAttribute, { attribute: a, owner: node });
        })
      );
    }
  });
\end{minted}

This is code is the a simplified version of the code that renders a
graph's nodes attributes into the tree-view-controller. This component
simply decides what attributes should be rendered into the tree.
\texttt{TreeNodeAttribute} is another component that will renders
different elements depending on what attribute is passed in.

Because the outputted html is only a function of its input it easy to
parse the scene-graph: 1. choose a graph node as the root 2. call the
node component with that graph node 3. if the graph node has child nodes
call the node component again with each child node and return their
return values wrapped in an element 4. if the graph node has no children
return an empty element

\subsubsection{Why the chosen approach will
work}\label{why-the-chosen-approach-will-work}

After trying different solutions it turned out the a functional solution
would also suit SceGraToo the best. First Chaplin (a backbone successor)
was used to implement SceGraToo, but soon the first version of it
suffocated under it's own complexity (probably also due to the
incompetence of its user - me). It turned out that MVC had serious flaws
when trying to use it to describe an ever changing declarative
scene-graph. The naive solution would be to observe the X3D node and
rerender the whole tree structure whenever it changed, that would also
mean to rerender every time an attribute is changed. To make that clear,
that means rerendering every time an object is moved with the mouse,
since the translation is an attribute. This thesis will not contain any
benchmarks proving that manipulating the DOM from javascript is slow,
rather than that it is left as an exercise to the reader to research it
if she wants. React solves this issue quite elegantly by creating the
dom structure in javascript and diffing it with the {[}DOM{]}, only
applying the minimum of changes to the DOM to realize the corresponding
result. That made it possible to use the X3D node of the DOM as the only
source of truth and minimize the state that needs to be kept to make the
tree-view-controller work.

A solution that handles state manually can work, but requires more
discipline and a more complicated mental model. The programmer has to
keep all side effects and cascading effects in mind when changing parts
of the program.
